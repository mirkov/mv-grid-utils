\documentclass[12pt]{article}%
\usepackage{chicago}
\usepackage{graphicx}
\usepackage{amsmath}%
\setcounter{MaxMatrixCols}{30}%
\usepackage{amsfonts}%
\usepackage{amssymb}
%TCIDATA{OutputFilter=latex2.dll}
%TCIDATA{Version=5.50.0.2953}
%TCIDATA{CSTFile=TTCA-BriefReport.cst}
%TCIDATA{Created=Thursday, March 18, 2010 15:41:57}
%TCIDATA{LastRevised=Thursday, March 18, 2010 15:43:00}
%TCIDATA{<META NAME="GraphicsSave" CONTENT="32">}
%TCIDATA{<META NAME="SaveForMode" CONTENT="1">}
%TCIDATA{BibliographyScheme=BibTeX}
%TCIDATA{<META NAME="DocumentShell" CONTENT="TTCA shells\TTCA-BriefReport">}
%BeginMSIPreambleData
\providecommand{\U}[1]{\protect\rule{.1in}{.1in}}
%EndMSIPreambleData
\begin{document}

\title{Title}
\author{Author}
\maketitle

\subsection{Numerical solution of the boundary value problem}

I first state the general formalism, and then specialize to the $\Psi_{0}$ and
$\Psi_{1}$.

The numerical solution proceeds by writing the second order derivative and the
equation on a grid of points $\xi_{i}$ in a very general form as%
\begin{equation}
\frac{c_{i-1}-2c_{i}+c_{i+1}}{\Delta^{2}}+f_{i}c_{i}+g_{i}+h_{i}=0
\label{eq:2nd-order-BVP-ODE-discretization}%
\end{equation}
where%
\begin{equation}
c_{i}=\Psi_{0,1}\left(  \xi_{i}\right)
\end{equation}
The functions $f$, $g$, and $h$ are in general%
\begin{align}
f_{i}  &  =f\left(  \Psi_{0}\left(  \xi_{i}\right)  ,\Psi_{1}\left(  \xi
_{i}\right)  ,\xi_{i}\right) \\
g_{i}  &  =g\left(  \Psi_{0}\left(  \xi_{i}\right)  ,\Psi_{1}\left(  \xi
_{i}\right)  ,\xi_{i}\right) \\
h_{i}  &  =h\left(  \xi_{i}\right)
\end{align}
The first of these, $f$, can be used for linear, or mildly non-linear problems
while $g$ for fully non-linear problems

Collecting terms%
\begin{equation}
-\frac{1}{\Delta^{2}}c_{i-1}+\left(  f_{i}+\frac{2}{\Delta^{2}}\right)
c_{i}-\frac{1}{\Delta^{2}}c_{i+1}+g_{i}+h_{i}=0
\end{equation}
we obtain a system of equations for the unknowns $c$.

We define a of $N$ equidistant points between $\xi_{0}$ and $\xi_{N-1}$. The
points are indexed
\begin{equation}
i=0,1,\ldots,N-1
\end{equation}
The value of $\Delta$ is
\begin{equation}
\Delta=\frac{\xi_{N-1}-\xi_{0}}{N-1}%
\end{equation}


Considering the boundary conditions from (\ref{eq:Psi1-equations}) we have the
following discretization%
\begin{align}
c_{0}  &  =c(\xi_{0})\\
-\frac{1}{\Delta^{2}}c_{0}+\left(  f_{1}+\frac{2}{\Delta^{2}}\right)
c_{1}-\frac{1}{\Delta^{2}}c_{2}  &  =g_{1}\\
-\frac{1}{\Delta^{2}}c_{i-1}+\left(  f_{i}+\frac{2}{\Delta^{2}}\right)
c_{i}-\frac{1}{\Delta^{2}}c_{i+1}  &  =g_{i}\\
-\frac{1}{\Delta^{2}}c_{N-3}+\left(  f_{N-2}+\frac{2}{\Delta^{2}}\right)
c_{N-2}-\frac{1}{\Delta^{2}}c_{N-1}  &  =g_{N-2}\\
c_{N-1}  &  =c\left(  \xi_{N-1}\right)
\end{align}


In terms of a tridiagonal matrix stored as a diagonal element D, superdiagonal
E and subdiagonal F, their contents are%
\begin{equation}
D_{0,1,\ldots i,\ldots N-2,N-1}=\left[
\begin{array}
[c]{c}%
1\\
\left(  f_{1}+2/\Delta^{2}\right) \\
\vdots\\
\left(  f_{i}+2/\Delta^{2}\right) \\
\vdots\\
\left(  f_{N-2}+2/\Delta^{2}\right) \\
1
\end{array}
\right]  \label{eq:Psi_1--D-vec}%
\end{equation}
%

\begin{equation}
E_{0,\ldots,N-2}=-\frac{1}{\Delta^{2}}\left[
\begin{array}
[c]{c}%
0\\
1\\
\vdots\\
1
\end{array}
\right]  \label{eq:Psi_1--Evec}%
\end{equation}
%

\begin{equation}
F_{0,\ldots,N-2}=-\frac{1}{\Delta^{2}}\left[
\begin{array}
[c]{c}%
1\\
\vdots\\
1\\
0
\end{array}
\right]  \label{eq:Psi_1--Fvec}%
\end{equation}
and%
\begin{equation}
B_{0,1,\ldots i,\ldots N-2,N-1}=\left[
\begin{array}
[c]{c}%
c_{0}\\
g_{1}\\
\vdots\\
g_{i}\\
\vdots\\
g_{N-2}\\
c_{N-1}%
\end{array}
\right]
\end{equation}
\bibliographystyle{CHICAGO}
\bibliography{CommonAbbrevs,JournalNameAbbrevs}



\end{document}