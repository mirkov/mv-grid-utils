\documentclass[12pt]{article}%
\usepackage{chicago}
\usepackage{graphicx}
\usepackage{amsmath}
\usepackage{amsfonts}
\usepackage{amssymb}%
\setcounter{MaxMatrixCols}{30}
%TCIDATA{OutputFilter=latex2.dll}
%TCIDATA{Version=5.50.0.2953}
%TCIDATA{CSTFile=TTCA-BriefReport.cst}
%TCIDATA{Created=Thursday, March 18, 2010 15:41:57}
%TCIDATA{LastRevised=Friday, June 18, 2010 10:02:19}
%TCIDATA{<META NAME="GraphicsSave" CONTENT="32">}
%TCIDATA{<META NAME="SaveForMode" CONTENT="1">}
%TCIDATA{BibliographyScheme=BibTeX}
%TCIDATA{<META NAME="DocumentShell" CONTENT="TTCA shells\TTCA-BriefReport">}
%BeginMSIPreambleData
\providecommand{\U}[1]{\protect\rule{.1in}{.1in}}
%EndMSIPreambleData
\begin{document}

\title{Title}
\author{Author}
\maketitle

\section{Numerical solution of the 2nd order boundary value problem}

\subsection{Setup of the tridiagonal system}

Second derivative approximation%
\begin{align}
y^{\prime}  & =\frac{c_{i}-c_{i-1}}{\Delta}\\
y^{\prime\prime}  & =\frac{\frac{c_{i+1}-c_{i}}{\Delta}-\frac{c_{i}-c_{i-1}%
}{\Delta}}{\Delta}\\
& =\frac{c_{i-1}-2c_{i}+c_{i+1}}{\Delta^{2}}%
\end{align}


The numerical solution proceeds by writing the second order derivative and the
equation on a grid of points $\xi_{i}$ in a very general form as%
\begin{equation}
\frac{c_{i-1}-2c_{i}+c_{i+1}}{\Delta^{2}}+a_{i}c_{i}+b_{i}=0
\label{eq:2nd-order-BVP-ODE-discretization}%
\end{equation}
where%
\begin{equation}
c_{i}=\Psi\left(  \xi_{i}\right)
\end{equation}
The functions $\alpha$ and $\beta$ are the values of functions $A$ and $B$ at
the node points%
\begin{align}
a_{i}  &  =A\left(  \Psi_{0}\left(  \xi_{i}\right)  ,\Psi_{1}\left(  \xi
_{i}\right)  ,\xi_{i}\right) \\
b_{i}  &  =B\left(  \Psi_{0}\left(  \xi_{i}\right)  ,\Psi_{1}\left(  \xi
_{i}\right)  ,\xi_{i}\right)
\end{align}


Collecting terms%
\begin{equation}
\frac{1}{\Delta^{2}}c_{i-1}+\left(  \alpha_{i}-\frac{2}{\Delta^{2}}\right)
c_{i}+\frac{1}{\Delta^{2}}c_{i+1}+\beta_{i}=0
\end{equation}
we obtain a system of equations for the unknowns $c_{i}$.

We define a of $N$ equidistant points between $\xi_{0}$ and $\xi_{N-1}$. The
points are indexed
\begin{equation}
i=0,1,\ldots,N-1
\end{equation}
The value of $\Delta$ is
\begin{equation}
\Delta=\frac{\xi_{N-1}-\xi_{0}}{N-1}%
\end{equation}


Considering the boundary conditions from (\ref{eq:Psi1-equations}) we have the
following discretization%
\begin{align}
c_{0} &  =c(\xi_{0})\\
\frac{1}{\Delta^{2}}c_{0}+\left(  \alpha_{1}-\frac{2}{\Delta^{2}}\right)
c_{1}+\frac{1}{\Delta^{2}}c_{2} &  =-\beta_{1}\\
\frac{1}{\Delta^{2}}c_{i-1}+\left(  \alpha_{i}-\frac{2}{\Delta^{2}}\right)
c_{i}+\frac{1}{\Delta^{2}}c_{i+1} &  =-\beta_{i}\\
\frac{1}{\Delta^{2}}c_{N-3}+\left(  \alpha_{N-2}-\frac{2}{\Delta^{2}}\right)
c_{N-2}+\frac{1}{\Delta^{2}}c_{N-1} &  =-\beta_{N-2}\\
c_{N-1} &  =c\left(  \xi_{N-1}\right)
\end{align}


In terms of a tridiagonal matrix stored as a diagonal element D, superdiagonal
E and subdiagonal F, their contents are%
\begin{equation}
D_{0,\ldots,N-1}=\left[
\begin{array}
[c]{c}%
1\\
\left(  \alpha_{1}-2/\Delta^{2}\right)  \\
\vdots\\
\left(  \alpha_{i}-2/\Delta^{2}\right)  \\
\vdots\\
\left(  \alpha_{N-2}-2/\Delta^{2}\right)  \\
1
\end{array}
\right]  \label{eq:Psi_1--D-vec}%
\end{equation}
%

\begin{equation}
E_{0,\ldots,N-2}=-\frac{1}{\Delta^{2}}\left[
\begin{array}
[c]{c}%
0\\
1\\
\vdots\\
1
\end{array}
\right]  \label{eq:Psi_1--Evec}%
\end{equation}
%

\begin{equation}
F_{0,\ldots,N-2}=-\frac{1}{\Delta^{2}}\left[
\begin{array}
[c]{c}%
1\\
\vdots\\
1\\
0
\end{array}
\right]  \label{eq:Psi_1--Fvec}%
\end{equation}
and%
\begin{equation}
B_{0,\ldots,N-1}=\left[
\begin{array}
[c]{c}%
c_{0}\\
\beta_{1}\\
\vdots\\
\beta_{i}\\
\vdots\\
\beta_{N-2}\\
c_{N-1}%
\end{array}
\right]
\end{equation}


\subsection{Sample solutions}

I consider problems of the form%
\begin{equation}
y^{\prime\prime}+fy=g
\end{equation}


\subsubsection{Linear solutions $f=0$, $g=0$}

The equation is%
\begin{equation}
y^{\prime\prime}=0
\end{equation}
which is integrated into%
\begin{equation}
y=ax+b
\end{equation}


\subsubsection{Higher order solutions for $g\neq0$}

For $g=1$ we get%
\begin{align}
y^{\prime}  &  =x+a\\
y  &  =\frac{x^{2}}{2}+ax+b
\end{align}


Boundary problem%
\begin{align}
y\left(  0\right)   &  =0\\
y\left(  5\right)   &  =5
\end{align}


From $y\left(  0\right)  =0$, $b=0$. Then at $x=5$%
\begin{equation}
5=\frac{25}{2}+5a
\end{equation}
It follows that%
\begin{align}
a  &  =\frac{5-12.5}{5}\\
&  =-\frac{7.5}{5}\\
&  =-1.5
\end{align}


For $\beta\left(  x\right)  =x^{n}$ we get%
\begin{align}
y^{\prime\prime}  &  =x^{n}\\
y^{\prime}  &  =\frac{x^{n+1}}{n+1}+a\\
y  &  =\frac{x^{n+2}}{n+2}+ax+b
\end{align}
For $b=0$ we get%
\begin{equation}
y=\frac{x^{n+2}}{n+2}+ax
\end{equation}


For the boundary problem we get%
\begin{align}
y_{1}  &  =\frac{x_{1}^{n+2}}{n+2}+ax_{1}\\
\frac{y_{1}}{x_{1}}  &  =\frac{x_{1}^{n+1}}{n+2}+a
\end{align}
But for programming reasons it is easier to keep the original version%
\begin{equation}
ax_{1}=y_{1}-\frac{x^{n+2}}{n+2}%
\end{equation}
or%
\begin{equation}
a=\frac{y_{1}-x_{1}^{n+2}/\left(  n+2\right)  }{x_{1}}%
\end{equation}


For $x_{1}=5$ we get%
\begin{align}
n  &  =0\\
a  &  =\frac{5-x_{1}^{n+2}/\left(  n+2\right)  }{x_{1}}\\
a  &  =-1.\,\allowbreak5\\
n  &  =1\\
a  &  =\frac{5-x_{1}^{n+2}/\left(  n+2\right)  }{x_{1}}\\
a  &  =-7.\,\allowbreak333\,3
\end{align}


For $g\left(  x\right)  =\cos\left(  kx\right)  $ we get%
\begin{align}
y^{\prime\prime}  &  =\cos\left(  kx\right) \\
y^{\prime}  &  =-\frac{1}{k}\sin\left(  kx\right)  +a\\
y  &  =-\frac{1}{k^{2}}\cos\left(  kx\right)  +ax+b
\end{align}
The LHS boundary condition imposes%
\begin{equation}
b=1/k^{2}%
\end{equation}
The RHS boundary imposes%
\begin{equation}
y_{1}=-\frac{1}{k^{2}}\cos\left(  kx_{1}\right)  +ax_{1}+1/k^{2}%
\end{equation}
from which we get%
\begin{equation}
a=\frac{y_{1}+\left(  \cos kx_{1}-1\right)  /k^{2}}{x_{1}}%
\end{equation}


The solution is%
\begin{align}
y  &  =-\frac{1}{k^{2}}\left(  \cos kx-1\right)  +\frac{y_{1}+\left(  \cos
kx_{1}-1\right)  /k^{2}}{x_{1}}x\\
&  =-\frac{\cos kx-1}{k^{2}}+\left(  y_{1}+\frac{\cos kx_{1}-1}{k^{2}}\right)
\frac{x}{x_{1}}%
\end{align}


\subsection{Solutions for $f\neq0$}

For $f=k^{2}$ we have%
\begin{equation}
y^{\prime\prime}-k^{2}y=0
\end{equation}%
\begin{align}
y  &  =A\exp\left(  -kx\right)  +B\exp\left(  kx\right) \\
&  =A\cosh\left(  kx\right)  +B\sinh\left(  kx\right)
\end{align}


For the two boundary conditions we get%
\begin{align}
y_{0}  &  =A\cosh\left(  kx_{0}\right)  +B\sinh\left(  kx_{0}\right) \\
y_{1}  &  =A\cosh\left(  kx_{1}\right)  +B\sinh\left(  kx_{1}\right)
\end{align}


The constants $A$ and $B$ are determined from%
\begin{equation}
\left[
\begin{array}
[c]{cc}%
\cosh\left(  kx_{0}\right)  & \sinh\left(  kx_{0}\right) \\
\cosh\left(  kx_{1}\right)  & \sinh\left(  kx_{1}\right)
\end{array}
\right]  \left[
\begin{array}
[c]{c}%
A\\
B
\end{array}
\right]  =\left[
\begin{array}
[c]{c}%
y_{0}\\
y_{1}%
\end{array}
\right]
\end{equation}
with the solution%
\begin{align}
\left[
\begin{array}
[c]{c}%
A\\
B
\end{array}
\right]   &  =\left[
\begin{array}
[c]{cc}%
\cosh\left(  kx_{0}\right)  & \sinh\left(  kx_{0}\right) \\
\cosh\left(  kx_{1}\right)  & \sinh\left(  kx_{1}\right)
\end{array}
\right]  ^{-1}\left[
\begin{array}
[c]{c}%
y_{0}\\
y_{1}%
\end{array}
\right] \\
&  =\frac{1}{\Delta}\left[
\begin{array}
[c]{c}%
y_{0}\sinh kx_{1}-y_{1}\sinh kx_{0}\\
y_{1}\cosh kx_{0}-y_{0}\cosh kx_{1}%
\end{array}
\right]  \allowbreak
\end{align}
$\allowbreak$and
\begin{equation}
\Delta=\cosh kx_{0}\sinh kx_{1}-\cosh kx_{1}\sinh kx_{0}%
\end{equation}
$\allowbreak$

For $f=-k^{2}$ we have%
\begin{equation}
y=A\cos\left(  kx\right)  +B\sin\left(  kx\right)
\end{equation}


For $f=1/x$ we have%
\begin{equation}
y^{\prime\prime}+x^{2}y=0
\end{equation}


Solution%
\begin{align}
y  &  =Ae^{\alpha x^{2}}\\
y^{\prime}  &  =A\alpha xe^{\alpha x^{2}}\\
y^{\prime\prime}  &  =A\alpha e^{\alpha x^{2}}+A\alpha^{2}x^{2}e^{\alpha
x^{2}}%
\end{align}
Substituting back we get%
\begin{equation}
A\alpha e^{\alpha x^{2}}+A\alpha^{2}x^{2}e^{\alpha x^{2}}+Ax^{2}e^{\alpha
x^{2}}=0
\end{equation}


\bibliographystyle{CHICAGO}
\bibliography{CommonAbbrevs,JournalNameAbbrevs}

\end{document}